%%%%%%%%%%%%%%%%%%%%%%%%%%%%%%%%%%%%%%%%%
% McMaster Masters/Doctoral Thesis
% LaTeX Template
% Version 2.2 (11/23/15)
%
% This template has been downloaded from:
% http://www.LaTeXTemplates.com
% Then subsequently from http://www.overleaf.com
%
% Version 2.0 major modifications by:
% Vel (vel@latextemplates.com)
%
% Original authors:
% Steven Gunn  (http://users.ecs.soton.ac.uk/srg/softwaretools/document/templates/)
% Sunil Patel (http://www.sunilpatel.co.uk/thesis-template/)
%
% Modified to McMaster format by Benjamin Furman (contact: https://www.xenben/com; Most up
% to date template at https://github.com/benjaminfurman/McMaster_Thesis_Template,
% occasionally updated on Overleaf template page)
%
% Modified for macdown by Antonio Paez; most up to date version at https://github.com/paezha/macdown
%
% License:
% CC BY-NC-SA 3.0 (http://creativecommons.org/licenses/by-nc-sa/3.0/)
%
%%%%%%%%%%%%%%%%%%%%%%%%%%%%%%%%%%%%%%%%%

%----------------------------------------------------------------------------------------
% DOCUMENT CONFIGURATIONS
%----------------------------------------------------------------------------------------

\documentclass[
11pt, % The default document font size, options: 10pt, 11pt, 12pt
oneside, % Two side (alternating margins) for binding by default, uncomment to switch to one side
english, % other languages available
singlespacing, % Single line spacing, alternatives: onehalfspacing or doublespacing
%draft, % Uncomment to enable draft mode (no pictures, no links, overfull hboxes indicated)
%nolistspacing, % If the document is onehalfspacing or doublespacing, uncomment this to set spacing in lists to single
%liststotoc, % Uncomment to add the list of figures/tables/etc to the table of contents
%toctotoc, % Uncomment to add the main table of contents to the table of contents
]{macthesis} % The class file specifying the document structure

%----------------------------------------------------------------------------------------
% Import packages here
%----------------------------------------------------------------------------------------
\usepackage[utf8]{inputenc} % Required for inputting international characters
\usepackage[T1]{fontenc} % Output font encoding for international characters
\usepackage{lastpage} % count pages
\usepackage{lmodern} % could change font type by calling a different package
\usepackage{lscape} % for landscaping pages
% New commands for landscape orientation
\newcommand{\blandscape}{\begin{landscape}}
\newcommand{\elandscape}{\end{landscape}}
%
\usepackage{siunitx} % for scientific units (micro-liter, etc)
\setcounter{tocdepth}{2} % so that only section and sub sections appear in Table of Contents. Remove or set depth to 3 to include sub-sub-sections

%----------------------------------------------------------------------------------------
% Define a blank page
%----------------------------------------------------------------------------------------
\def\blankpage{%
      \clearpage%
      \thispagestyle{empty}%
      \addtocounter{page}{-1}%
      \null%
      \clearpage}

%----------------------------------------------------------------------------------------
% Define a tight list
%----------------------------------------------------------------------------------------
\def\tightlist{}

%----------------------------------------------------------------------------------------
%	Highlight Code Chunks
%----------------------------------------------------------------------------------------
  \usepackage{color}
  \usepackage{fancyvrb}
  \newcommand{\VerbBar}{|}
  \newcommand{\VERB}{\Verb[commandchars=\\\{\}]}
  \DefineVerbatimEnvironment{Highlighting}{Verbatim}{commandchars=\\\{\}}
  % Add ',fontsize=\small' for more characters per line
  \usepackage{framed}
  \definecolor{shadecolor}{RGB}{248,248,248}
  \newenvironment{Shaded}{\begin{snugshade}}{\end{snugshade}}
  \newcommand{\AlertTok}[1]{\textcolor[rgb]{0.94,0.16,0.16}{#1}}
  \newcommand{\AnnotationTok}[1]{\textcolor[rgb]{0.56,0.35,0.01}{\textbf{\textit{#1}}}}
  \newcommand{\AttributeTok}[1]{\textcolor[rgb]{0.13,0.29,0.53}{#1}}
  \newcommand{\BaseNTok}[1]{\textcolor[rgb]{0.00,0.00,0.81}{#1}}
  \newcommand{\BuiltInTok}[1]{#1}
  \newcommand{\CharTok}[1]{\textcolor[rgb]{0.31,0.60,0.02}{#1}}
  \newcommand{\CommentTok}[1]{\textcolor[rgb]{0.56,0.35,0.01}{\textit{#1}}}
  \newcommand{\CommentVarTok}[1]{\textcolor[rgb]{0.56,0.35,0.01}{\textbf{\textit{#1}}}}
  \newcommand{\ConstantTok}[1]{\textcolor[rgb]{0.56,0.35,0.01}{#1}}
  \newcommand{\ControlFlowTok}[1]{\textcolor[rgb]{0.13,0.29,0.53}{\textbf{#1}}}
  \newcommand{\DataTypeTok}[1]{\textcolor[rgb]{0.13,0.29,0.53}{#1}}
  \newcommand{\DecValTok}[1]{\textcolor[rgb]{0.00,0.00,0.81}{#1}}
  \newcommand{\DocumentationTok}[1]{\textcolor[rgb]{0.56,0.35,0.01}{\textbf{\textit{#1}}}}
  \newcommand{\ErrorTok}[1]{\textcolor[rgb]{0.64,0.00,0.00}{\textbf{#1}}}
  \newcommand{\ExtensionTok}[1]{#1}
  \newcommand{\FloatTok}[1]{\textcolor[rgb]{0.00,0.00,0.81}{#1}}
  \newcommand{\FunctionTok}[1]{\textcolor[rgb]{0.13,0.29,0.53}{\textbf{#1}}}
  \newcommand{\ImportTok}[1]{#1}
  \newcommand{\InformationTok}[1]{\textcolor[rgb]{0.56,0.35,0.01}{\textbf{\textit{#1}}}}
  \newcommand{\KeywordTok}[1]{\textcolor[rgb]{0.13,0.29,0.53}{\textbf{#1}}}
  \newcommand{\NormalTok}[1]{#1}
  \newcommand{\OperatorTok}[1]{\textcolor[rgb]{0.81,0.36,0.00}{\textbf{#1}}}
  \newcommand{\OtherTok}[1]{\textcolor[rgb]{0.56,0.35,0.01}{#1}}
  \newcommand{\PreprocessorTok}[1]{\textcolor[rgb]{0.56,0.35,0.01}{\textit{#1}}}
  \newcommand{\RegionMarkerTok}[1]{#1}
  \newcommand{\SpecialCharTok}[1]{\textcolor[rgb]{0.81,0.36,0.00}{\textbf{#1}}}
  \newcommand{\SpecialStringTok}[1]{\textcolor[rgb]{0.31,0.60,0.02}{#1}}
  \newcommand{\StringTok}[1]{\textcolor[rgb]{0.31,0.60,0.02}{#1}}
  \newcommand{\VariableTok}[1]{\textcolor[rgb]{0.00,0.00,0.00}{#1}}
  \newcommand{\VerbatimStringTok}[1]{\textcolor[rgb]{0.31,0.60,0.02}{#1}}
  \newcommand{\WarningTok}[1]{\textcolor[rgb]{0.56,0.35,0.01}{\textbf{\textit{#1}}}}

%----------------------------------------------------------------------------------------
% Handling Citations
%----------------------------------------------------------------------------------------

% definitions for citeproc citations
\NewDocumentCommand\citeproctext{}{}
\NewDocumentCommand\citeproc{mm}{%
\begingroup\def\citeproctext{#2}\cite{#1}\endgroup}
\makeatletter
% allow citations to break across lines
\let\@cite@ofmt\@firstofone
% avoid brackets around text for \cite:
\def\@biblabel#1{}
\def\@cite#1#2{{#1\if@tempswa , #2\fi}}
\makeatother
\newlength{\cslhangindent}
\setlength{\cslhangindent}{1.5em}
\newlength{\csllabelwidth}
\setlength{\csllabelwidth}{3em}
\newenvironment{CSLReferences}[2] % #1 hanging-indent, #2 entry-spacing
{\begin{list}{}{%
	\setlength{\itemindent}{0pt}
	\setlength{\leftmargin}{0pt}
	\setlength{\parsep}{0pt}
	% turn on hanging indent if param 1 is 1
	\ifodd #1
	\setlength{\leftmargin}{\cslhangindent}
	\setlength{\itemindent}{-1\cslhangindent}
	\fi
	% set entry spacing
	\setlength{\itemsep}{#2\baselineskip}}}
{\end{list}}
\usepackage{calc}
\newcommand{\CSLBlock}[1]{\hfill\break\parbox[t]{\linewidth}{\strut\ignorespaces#1\strut}}
\newcommand{\CSLLeftMargin}[1]{\parbox[t]{\csllabelwidth}{\strut#1\strut}}
\newcommand{\CSLRightInline}[1]{\parbox[t]{\linewidth - \csllabelwidth}{\strut#1\strut}}
\newcommand{\CSLIndent}[1]{\hspace{\cslhangindent}#1}


%----------------------------------------------------------------------------------------
% Collect all your header information from the chapters here, things like acronyms, custom commands, necessary packages, etc.
%----------------------------------------------------------------------------------------
\usepackage{parskip} %this will put spaces between paragraphs
\setlength{\parindent}{15pt} % this will create and indent on all but the first paragraph of each section.
% should maybe change to glossaries package
\usepackage{acro}
\DeclareAcronym{est}{
	short = EST,
	long  = expressed sequence tags
}

\DeclareAcronym{Xl}{
	short = \textit{X.~laevis},
	long  = \textit{Xenopus~laevis}
}
\DeclareAcronym{Xg}{
	short = \textit{X.~gilli},
	long  = \textit{Xenopus~gilli}
}

\usepackage{etoolbox}
\preto\chapter{\acresetall} % resets acronyms for each chapter

\usepackage{xspace} %helps spacing with custom commands.
\newcommand{\oddname}{{\sc SoME goOfY LonG ThiNg With an AwkWarD NAme}\xspace}


\usepackage{pgfplotstable} % a much better way to handle tables
\pgfplotsset{compat=1.12}

% \usepackage{float} % if you need to demand figure/table placement, then this will allow you to use [H], which demands a figure placement. Beware, making LaTeX do things it doesn't want may lead to oddities.


%%%%
% LINK COLORS
% You can control the link colors at the end of the McMasterThesis.cls file. There is also a true/false option there to turn off all link colors.
%%%%


%----------------------------------------------------------------------------------------
%	THESIS INFORMATION
%----------------------------------------------------------------------------------------

\title{00-thesis-draft}
%\thesistitle{Thesis Title} % Your thesis title, print it elsewhere with \ttitle
\author{You R. Name}
%\author{John \textsc{Smith}} % Your name, print it elsewhere with \authorname
\bdegree{B.Sc.}
\mdegree{}
%Previous degrees % print it elsewhere with \bdeg and \mdeg
\date{2025-04-05}
% The month and year that you submit your FINAL draft TO THE LIBRARY (May or December)
\university{McMaster University}
%\university{\href{http://www.mcmaster.ca/}{McMaster University}} % Your university's name and URL, print it elsewhere with \univname
%\division{}
\faculty{Faculty of Science} % Your faculty's name and URL, print it elsewhere with \facname
\department{School of Earth, Environment and Society} % Your department's name and URL, print it elsewhere with \deptname
\subject{Geography} % Your subject area, print it elsewhere with \subjectname
%\group{\href{http://researchgroup.university.com}{Research Group Name}} % Your research group's name and URL, print it elsewhere with \groupname
\supervisor{Antonio Paez}
%\supervisor{Dr. Jane \textsc{Smith}} % Your supervisor's name, print it elsewhere with \supname
\examiner{} % Your examiner's name, print it elsewhere with \examname
\degree{Doctor of Philosophy}
%\degree{Doctor of Philosophy} % Your degree name, print it elsewhere with \degreename
\addresses{} % Your address, print it elsewhere with \addressname
\keywords{} % Keywords for your thesis, print it elsewhere with \keywordnames


% this sets up hyperlinks
\hypersetup{pdftitle=\ttitle} % Set the PDF's title to your title
\hypersetup{pdfauthor=\authorname} % Set the PDF's author to your name
\hypersetup{pdfkeywords=\keywordnames} % Set the PDF's keywords to your keywords

\begin{document}
\sloppy

\frontmatter % Use roman page numbering style (i, ii, iii, iv...) for the pre-content pages

\pagestyle{plain} % Default to the plain heading style until the thesis style is called for the body content

%----------------------------------------------------------------------------------------
%	Half Title (lay title)
%----------------------------------------------------------------------------------------
%\begin{halftitle} % could not get this environment working
%\vspace*{\fill}
\vspace{6cm}
\begin{center}
\ttitle
\end{center}
%\vspace*{\fill}
\pagenumbering{gobble} % leave this here, McMaster doesn't want this page numbered
%\end{halftitle}
\clearpage

%----------------------------------------------------------------------------------------
%	TITLE PAGE
%----------------------------------------------------------------------------------------
\pagenumbering{gobble}
\begin{center}

\vfill
\textsc{\Large \ttitle} \\

\vfill
{By \authorname\, \bdeg }


 \vfill
{\large \textit{A Thesis Submitted to the School of Graduate Studies in the Partial Fulfillment of the Requirements for the Degree \degreename}}\\

\vfill
{\large \univname\, \copyright\, Copyright by \authorname\, \today}\\[4cm] % replace \today with the submission date

\end{center}
\blankpage
\clearpage

%----------------------------------------------------------------------------------------
%	QUOTATION PAGE
%----------------------------------------------------------------------------------------

\vspace*{0.2\textheight}

\noindent{\itshape You may hate gravity, but gravity does not care}\bigbreak

\hfill\textemdash Albert Einstein

\blankpage
\clearpage

%%%%%%%%%%%%%%%%%%%%%%%%%%%
%%%%%%%%%%%%%%%%%%%%%%%%%%%
% optional page stuff
%----------------------------------------------------------------------------------------
% can do physical constraints and symbols pages, see the original thesis example on overleaf if you want to include them at https://www.overleaf.com/latex/templates/template-for-a-masters-slash-doctoral-thesis/mkzrzktcbzfl#.VlPeicorpE4
%----------------------------------------------------------------------------------------

%----------------------------------------------------------------------------------------
%	DEDICATION
%----------------------------------------------------------------------------------------

    You can have a dedication here if you wish.

\blankpage
\clearpage


%----------------------------------------------------------------------------------------
%	Descriptive note numbered ii
%----------------------------------------------------------------------------------------
% Need to add below info
\newpage
\pagenumbering{roman} % leave to turn numbering back on
\setcounter{page}{2} % leave here to make this page numbered ii, a Grad School requirement

\noindent % stops indent on next line
\univname \\
\degreename\, (\the\year) \\
Hamilton, Ontario (\deptname) \\[1.5cm]
TITLE: \ttitle \\
AUTHOR: \authorname\,  %list previous degrees
(\univname)  \\
SUPERVISOR: \supname\, \\
NUMBER OF PAGES: \pageref{lastoffront}, \pageref{LastPage}  % put in iv and number

\clearpage

%----------------------------------------------------------------------------------------
%	Lay abstract number iii
%----------------------------------------------------------------------------------------
% not actually included in most theses, though requested by the GSA
% uncomment below lines if you want to include one
\section*{Lay Abstract}
  The lay abstract must be 150 words or less.

  It must explain the key goals and contributions of the thesis in lay terms that are accessible to the general public.
\blankpage
\clearpage


%----------------------------------------------------------------------------------------
%	ABSTRACT PAGE number iv
%----------------------------------------------------------------------------------------

\section*{\Huge Abstract}
\addchaptertocentry{\abstractname}
% Type your abstract here.
This is the abstract.

I can write a really long abstract.
\blankpage
\clearpage

%----------------------------------------------------------------------------------------
%	ACKNOWLEDGEMENTS
%----------------------------------------------------------------------------------------

  \begin{acknowledgements}
  \addchaptertocentry{\acknowledgementname} % Add the acknowledgments to the table of contents
    I want to thank a few people.

    This includes my friends.
  \end{acknowledgements}
\blankpage
\clearpage

%----------------------------------------------------------------------------------------
%	LIST OF CONTENTS/FIGURES/TABLES PAGES
%----------------------------------------------------------------------------------------

\tableofcontents % Prints the main table of contents

\listoffigures % Prints the list of figures

\listoftables % Prints the list of tables

%----------------------------------------------------------------------------------------
%	ABBREVIATIONS
%----------------------------------------------------------------------------------------
% many theses don't use this section, as it will be declared at first use and again each chapter. Uncomment these four lines to activate if you want
%\clearpage
%\section*{\Huge Acronyms}
%\addchaptertocentry{Acronyms}
%\printacronyms[name] % name without an option stops the header

%----------------------------------------------------------------------------------------
%	DECLARATION PAGE
%----------------------------------------------------------------------------------------

\begin{declaration}
\addchaptertocentry{\authorshipname}

\noindent I, \authorname, declare that this thesis titled, \emph{\ttitle} and the work presented in it are my own. I confirm that:

I did most of the research.

Also the writting.

Sometimes I cried.

But mostly I had fun.

\end{declaration}


%----------------------------------------------------------------------------------------
% The following bit is just here to make sure we end up on a new page and get the total number of roman numeral
\label{lastoffront}
\clearpage
% make sure this command is on the last of your frontmatter pages, i.e. only this command, a \clearpage then \mainmatter
% should be fine without modification
%----------------------------------------------------------------------------------------

%----------------------------------------------------------------------------------------
%	THESIS MAIN BODY
%----------------------------------------------------------------------------------------

\mainmatter % here the regular arabic numbering starts
\pagestyle{thesis}
\chapter*{Preface}\label{preface}
\addcontentsline{toc}{chapter}{Preface}

Welcome to the \emph{R Markdown} thesis template. This template is based on (and in many places copied directly from) the Reed College LaTeX template, but hopefully it will provide a nicer interface for those that have never used TeX or LaTeX before. Using \emph{R Markdown} will also allow you to easily keep track of your analyses in \textbf{R} chunks of code, with the resulting plots and output included as well. The hope is this \emph{R Markdown} template gets you in the habit of doing reproducible research, which benefits you long-term as a researcher, but also will greatly help anyone that is trying to reproduce or build onto your results down the road.

Hopefully, you won't have much of a learning period to go through and you will reap the benefits of a nicely formatted thesis. The use of LaTeX in combination with \emph{Markdown} is more consistent than the output of a word processor, much less prone to corruption or crashing, and the resulting file is smaller than a Word file. While you may have never had problems using Word in the past, your thesis is likely going to be about twice as large and complex as anything you've written before, taxing Word's capabilities. After working with \emph{Markdown} and \textbf{R} together for a few weeks, we are confident this will be your reporting style of choice going forward.

\textbf{Why use it?}

\emph{R Markdown} creates a simple and straightforward way to interface with the beauty of LaTeX. Packages have been written in \textbf{R} to work directly with LaTeX to produce nicely formatting tables and paragraphs. In addition to creating a user friendly interface to LaTeX, \emph{R Markdown} also allows you to read in your data, to analyze it and to visualize it using \textbf{R} functions, and also to provide the documentation and commentary on the results of your project. Further, it allows for \textbf{R} results to be passed inline to the commentary of your results. You'll see more on this later.

\textbf{Who should use it?}

Anyone who needs to use data analysis, math, tables, a lot of figures, complex cross-references, or who just cares about the final appearance of their document should use \emph{R Markdown}. Of particular use should be anyone in the sciences, but the user-friendly nature of \emph{Markdown} and its ability to keep track of and easily include figures, automatically generate a table of contents, index, references, table of figures, etc. should make it of great benefit to nearly anyone writing a thesis project.

\textbf{For additional help with bookdown}

Please visit \href{https://bookdown.org/yihui/bookdown/}{the free online bookdown reference guide}.

\begin{Shaded}
\begin{Highlighting}[]
\ControlFlowTok{if}\NormalTok{ (}\SpecialCharTok{!}\FunctionTok{require}\NormalTok{(}\StringTok{"remotes"}\NormalTok{)) }
  \FunctionTok{install.packages}\NormalTok{(}\StringTok{"remotes"}\NormalTok{, }
                   \AttributeTok{repos =} \StringTok{"https://cran.rstudio.org"}\NormalTok{)}
\ControlFlowTok{if}\NormalTok{ (}\SpecialCharTok{!}\FunctionTok{require}\NormalTok{(}\StringTok{"bookdown"}\NormalTok{)) }
  \FunctionTok{install.packages}\NormalTok{(}\StringTok{"bookdown"}\NormalTok{, }
                   \AttributeTok{repos =} \StringTok{"https://cran.rstudio.org"}\NormalTok{)}
\end{Highlighting}
\end{Shaded}

\begin{verbatim}
Loading required package: bookdown
\end{verbatim}

\begin{Shaded}
\begin{Highlighting}[]
\NormalTok{remotes}\SpecialCharTok{::}\FunctionTok{install\_github}\NormalTok{(}\StringTok{"paezha/macdown"}\NormalTok{)}
\end{Highlighting}
\end{Shaded}

\begin{verbatim}
Using GitHub PAT from the git credential store.
\end{verbatim}

\begin{verbatim}
Skipping install of 'macdown' from a github remote, the SHA1 (4d47b0a2) has not changed since last install.
  Use `force = TRUE` to force installation
\end{verbatim}

\chapter{R Markdown Basics}\label{rmd-basics}

Here is a brief introduction into using \emph{R Markdown}. \emph{Markdown} is a simple formatting syntax for authoring HTML, PDF, and MS Word documents. \emph{R Markdown} provides the flexibility of \emph{Markdown} with the implementation of \textbf{R} input and output. For more details on using \emph{R Markdown} see \url{https://rmarkdown.rstudio.com}.

Be careful with your spacing in \emph{Markdown} documents. While whitespace largely is ignored, it does at times give \emph{Markdown} signals as to how to proceed. As a habit, try to keep everything left aligned whenever possible, especially as you type a new paragraph. In other words, there is no need to indent basic text in the Rmd document (in fact, it might cause your text to do funny things if you do).

Here is a reference to Angel (2000).

\chapter{Writing}\label{writing}

You can use \texttt{R} code in your document. For example:

`\texttt{markdown}\{r ch1-load-packages, message=FALSE\}
plot(cars)

\begin{verbatim}



Naming the code chunks is convenient for navigating your document. You can use chunk options to control what the code does and how it is displayed. See Yihui Xie's documentation about chunk options https://yihui.org/knitr/options/. In the chunk above `message=FALSE` forces the chunk to not display messages when the packages are loaded.

Of course, unless you are explicitly illustrating/discussing the code, you might not want the code to appear in the thesis! As another example, a chunk would _not_ be displayed _at all_ in the output document if `echo=FALSE`.



You can create elegant figures and tables using `R` and displaying them in your thesis.

Figure \ref{fig:ch1-scatterplot} is an example of a figure generated using the package `ggplot`.

\begin{figure}
\includegraphics[width=0.8\linewidth]{thesis_files/figure-latex/ch1-scatterplot-1} \caption{\label{fig:ch1-scatterplot}Example of a scatterplot}\label{fig:ch1-scatterplot}
\end{figure}

Table \ref{tab:summary-statistics} is an example of a table created using the package `kableExtra`.

\begin{table}

\caption{\label{tab:ch1-table}\label{tab:summary-statistics}Example of a table with summary statistics}
\centering
\begin{tabular}[t]{llll}
\toprule
  &     displ &      year &      cyl\\
\midrule
 & Min.   :1.600 & Min.   :1999 & Min.   :4.000\\
 & 1st Qu.:2.400 & 1st Qu.:1999 & 1st Qu.:4.000\\
 & Median :3.300 & Median :2004 & Median :6.000\\
 & Mean   :3.472 & Mean   :2004 & Mean   :5.889\\
 & 3rd Qu.:4.600 & 3rd Qu.:2008 & 3rd Qu.:8.000\\
\addlinespace
 & Max.   :7.000 & Max.   :2008 & Max.   :8.000\\
\bottomrule
\end{tabular}
\end{table}

Table \ref{tab:summary-statistics-2} is the same as Table \ref{tab:summary-statistics}, but placed in landscape orientation. Landscape orientation is useful for wide tables or for large figures.

<!-- To use landscape orientation start a new page with \newpage and begin the landscape with \begin{landscape}-->

<!-- You can have one or multiple pages in landscape, and when you are ready to switch to portrait orientation remember to end the landscape with \end{landscape} -->

\newpage
\blandscape
 

\begin{table}

\caption{\label{tab:ch1-table-2}\label{tab:summary-statistics-2}Example of a table in a landscape page}
\centering
\begin{tabular}[t]{llll}
\toprule
  &     displ &      year &      cyl\\
\midrule
 & Min.   :1.600 & Min.   :1999 & Min.   :4.000\\
 & 1st Qu.:2.400 & 1st Qu.:1999 & 1st Qu.:4.000\\
 & Median :3.300 & Median :2004 & Median :6.000\\
 & Mean   :3.472 & Mean   :2004 & Mean   :5.889\\
 & 3rd Qu.:4.600 & 3rd Qu.:2008 & 3rd Qu.:8.000\\
\addlinespace
 & Max.   :7.000 & Max.   :2008 & Max.   :8.000\\
\bottomrule
\end{tabular}
\end{table}

\elandscape
\newpage

<!--chapter:end:01-chap1.Rmd-->

# Mathematics and Science {#math-sci}

<!-- Required to number equations in HTML files -->
<script type="text/x-mathjax-config">
MathJax.Hub.Config({
  TeX: { equationNumbers: { autoNumber: "AMS" } }
});
</script>

## Math

\TeX\ is the best way to typeset mathematics. Donald Knuth designed \TeX\ when he got frustrated at how long it was taking the typesetters to finish his book, which contained a lot of mathematics.  One nice feature of _R Markdown_ is its ability to read LaTeX code directly.

If you are doing a thesis that will involve lots of math, you will want to read the following section which has been commented out. If you're not going to use math, skip over or delete this next commented section.


<!-- MATH and PHYSICS majors: Uncomment the following section -->
$$
\sum_{j=1}^n (\delta\theta_j)^2 \leq {{\beta_i^2}\over{\delta_i^2 + \rho_i^2}}
\left[ 2\rho_i^2 + {\delta_i^2\beta_i^2\over{\delta_i^2 + \rho_i^2}} \right] \equiv \omega_i^2
$$

From Informational Dynamics, we have the following (Dave Braden):
<!--

After _n_ such encounters the posterior density for $\theta$ is

$$
\pi(\theta|X_1< y_1,\dots,X_n<y_n) \varpropto \pi(\theta) \prod_{i=1}^n\int_{-\infty}^{y_i}
   \exp\left(-{(x-\theta)^2\over{2\sigma^2}}\right)\ dx
$$

Another equation:

$$
\det\left|\,\begin{matrix}%
c_0&c_1\hfill&c_2\hfill&\ldots&c_n\hfill\cr
c_1&c_2\hfill&c_3\hfill&\ldots&c_{n+1}\hfill\cr
c_2&c_3\hfill&c_4\hfill&\ldots&c_{n+2}\hfill\cr
\,\vdots\hfill&\,\vdots\hfill&
  \,\vdots\hfill&&\,\vdots\hfill\cr
c_n&c_{n+1}\hfill&c_{n+2}\hfill&\ldots&c_{2n}\hfill\cr
\end{matrix}\right|>0
$$


Lapidus and Pindar, Numerical Solution of Partial Differential Equations in Science and
Engineering.  Page 54

$$
\int_t\left\{\sum_{j=1}^3 T_j \left({d\phi_j\over dt}+k\phi_j\right)-kT_e\right\}w_i(t)\ dt=0,
   \qquad\quad i=1,2,3.
$$

L\&P  Galerkin method weighting functions.  Page 55

$$
\sum_{j=1}^3 T_j\int_0^1\left\{{d\phi_j\over dt} + k\phi_j\right\} \phi_i\ dt
   = \int_{0}^1k\,T_e\phi_idt, \qquad i=1,2,3 
$$

Another L\&P (p145)

$$
\int_{-1}^1\!\int_{-1}^1\!\int_{-1}^1 f\big(\xi,\eta,\zeta\big)
   = \sum_{k=1}^n\sum_{j=1}^n\sum_{i=1}^n w_i w_j w_k f\big( \xi,\eta,\zeta\big).
$$

Another L\&P (p126)

$$
\int_{A_e} (\,\cdot\,) dx dy = \int_{-1}^1\!\int_{-1}^1 (\,\cdot\,) \det[J] d\xi d\eta.
$$

\begin{equation}
  \mathrm{C_6H_{12}O_6  + 6O_2} \longrightarrow \mathrm{6CO_2 + 6H_2O}
  \label{eq:reaction}
\end{equation}

We can reference this combustion of glucose reaction via Equation \\eqref{eq:reaction}.

### Other examples of reactions

$\mathrm{NH_4Cl_{(s)}}$ $\rightleftharpoons$ $\mathrm{NH_{3(g)}+HCl_{(g)}}$

\noindent $\mathrm{MeCH_2Br + Mg}$ $\xrightarrow[below]{above}$ $\mathrm{MeCH_2\bullet Mg \bullet Br}$

-->

## Physics

Many of the symbols you will need can be found on the math page <https://web.reed.edu/cis/help/latex/math.html> and the Comprehensive LaTeX Symbol Guide (<https://mirror.utexas.edu/ctan/info/symbols/comprehensive/symbols-letter.pdf>).

## Biology

You will probably find the resources at <https://www.lecb.ncifcrf.gov/~toms/latex.html> helpful, particularly the links to bsts for various journals. You may also be interested in TeXShade for nucleotide typesetting (<https://homepages.uni-tuebingen.de/beitz/txe.html>).  Be sure to read the proceeding chapter on graphics and tables.


<!--chapter:end:02-chap2.Rmd-->

---
title: "02-methods"
output: pdf_document
date: "2025-04-05"
---


``` r
library(knitr)
library(markdown)
\end{verbatim}

\begin{Shaded}
\begin{Highlighting}[]
\FunctionTok{print}\NormalTok{(}\StringTok{"this is a test file"}\NormalTok{)}
\end{Highlighting}
\end{Shaded}

\begin{verbatim}
[1] "this is a test file"
\end{verbatim}

\chapter*{Conclusion}\label{conclusion}
\addcontentsline{toc}{chapter}{Conclusion}

If we don't want Conclusion to have a chapter number next to it, we can add the \texttt{\{-\}} attribute.

\textbf{More info}

And here's some other random info: the first paragraph after a chapter title or section head \emph{shouldn't be} indented, because indents are to tell the reader that you're starting a new paragraph. Since that's obvious after a chapter or section title, proper typesetting doesn't add an indent there.

\backmatter

\chapter*{References}\label{references}
\addcontentsline{toc}{chapter}{References}

\markboth{References}{References}

\noindent

\setlength{\parindent}{-0.20in}
\setlength{\leftskip}{0.20in}
\setlength{\parskip}{8pt}

\phantomsection\label{refs}
\begin{CSLReferences}{1}{0}
\bibitem[\citeproctext]{ref-angel2000}
Angel, E. (2000). \emph{Interactive computer graphics : A top-down approach with OpenGL}. Boston, MA: Addison Wesley Longman.

\bibitem[\citeproctext]{ref-angel2001}
Angel, E. (2001a). \emph{Batch-file computer graphics : A bottom-up approach with QuickTime}. Boston, MA: Wesley Addison Longman.

\bibitem[\citeproctext]{ref-angel2002a}
Angel, E. (2001b). \emph{Test second book by angel}. Boston, MA: Wesley Addison Longman.

\end{CSLReferences}

\end{document}
